\documentclass[nojss]{jss}

%%%%%%%%%%%%%%%%%%%%%%%%%%%%%%
%% declarations for jss.cls %%%%%%%%%%%%%%%%%%%%%%%%%%%%%%%%%%%%%%%%%%
%%%%%%%%%%%%%%%%%%%%%%%%%%%%%%

%% almost as usual
\author{Gary King\\Harvard University \And 
        Christopher Lucas\\Harvard University \And 
        Richard Nielsen\\MIT}
\title{\pkg{MatchingFrontier}: Computing the Balance-Sample Size Frontier in Matching Methods}

%% for pretty printing and a nice hypersummary also set:
\Plainauthor{Gary King, Christopher Lucas, Richard Nielsen} %% comma-separated
\Plaintitle{MatchingFrontier: Computing the Balance-Sample Size Frontier in Matching Methods} %% without formatting
\Shorttitle{\pkg{MatchingFrontier}} %%  short title (if necessary)

%% an abstract and keywords
\Abstract{ \pkg{MatchingFrontier} implements the methods described in
  King, Lucas, and Nielsen (n.d.) for optimizing both balance and 
  sample size. \pkg{MatchingFrontier} supports the computation
  of frontiers for both continuous and discrete metrics, and
  also provides functions for visualizing the frontier. Additional
  functionality to be added in the coming months.}

\Keywords{\proglang{R}, matching, frontier, Mahalanobis, L1}
\Plainkeywords{R, matching, frontier, Mahalanobis, L1} %% without formatting
\Address{
  Gary King\\
  Department of Government\\
  Harvard University\\
  1737 Cambridge St, Cambridge, MA, USA\\
  E-mail: \href{mailto:king@harvard.edu}{king@harvard.edu}\\
  URL: \href{http://gking.harvard.edu/}{http://gking.harvard.edu/}\\

  Christopher Lucas\\
  Department of Government\\
  Harvard University\\
  1737 Cambridge St, Cambridge, MA, USA\\
  E-mail: \href{mailto:clucas@fas.harvard.edu}{clucas@fas.harvard.edu}\\
  URL: \href{http://christopherlucas.org/}{christopherlucas.org}\\

  Richard Nielsen\\
  Department of Political Science\\
  Massachusetts Institute of Technology\\
  77 Massachusetts Avenue, Cambridge, MA, USA\\
  E-mail: \href{mailto:rnielsen@mit.edu}{rnielsen@mit.edu}\\
  URL: \href{http://www.mit.edu/~rnielsen/index.htm}{http://www.mit.edu/~rnielsen/index.htm}

}
%% It is also possible to add a telephone and fax number
%% before the e-mail in the following format:
%% Telephone: +43/512/507-7103
%% Fax: +43/512/507-2851

%% for those who use Sweave please include the following line (with % symbols):
%% need no \usepackage{Sweave.sty}

%% end of declarations %%%%%%%%%%%%%%%%%%%%%%%%%%%%%%%%%%%%%%%%%%%%%%%


\begin{document}

%% include your article here, just as usual
%% Note that you should use the \pkg{}, \proglang{} and \code{} commands.

\section[Introduction]{About \pkg{MatchingFrontier}}

\section[Usage]{Installation}

You can install the latest stable version of \pkg{MatchingFrontier}
with \code{install.packages(`MatchingFrontier', repos =
  `http://cran.us.r-project.org')}.  If you'd prefer to use the
development version, you may do so by installing the package from the
public-facing Github repository. We recommend using the \pkg{devtools}
package to do so, as follows.

\begin{Code}
  library(devtools)
  install_github('MatchingFrontier')
\end{Code} 

\section[Usage]{Using \pkg{MatchingFrontier}}

At present, the \pkg{MatchingFrontier} namespace includes just a few
functions (feedback on additional functionality is greatly
appreciated).  Those functions support the computation of the
frontier, the estimation of effects along it, plotting functions, and
support for exporting optimal data sets on the balance - sample size
frontier. We will illustrate the use of \pkg{MatchingFrontier} 
with the commonly used Lalonde data, which is included in the package.

The user must first create the frontier. To do so, use the \code{makeFrontier()}
function, as follows. 

\begin{CodeChunk}
\begin{CodeInput}
  data('lalonde')
  match.on <- colnames(lalonde)[!(colnames(lalonde) %in% c('re78', 'treat'))]
  my.frontier <- makeFrontier(dataset = lalonde, 
                              treatment = 'treat', 
                              outcome = 're78', 
                              match.on = match.on)
\end{CodeInput}
\end{CodeChunk}

\code{match.on} is a character vector holding the variable names that
the user wishes to match on. In the Lalonde data, \code{re78} is the
outcome and \code{treat} is the treatment, so we exclude those
variable names from the character vector.

By default, \code{makeFrontier()} calculates the frontier for the
Average Mahalanobis Distance metric, primarily because this is the
fastest at present. The quantity of interest is the \emph{feasible
  sample average treatment effect on the treated} or FSATT. Weights
are later computed in the estimation stage. The full function is as
follows.  For complete information, see the help file in the package
with \code{?makeFrontier}.

\begin{CodeChunk}
\begin{CodeInput}
  makeFrontier(dataset, 
              treatment, 
              outcome, 
              match.on, QOI = 'FSATT',
              metric = 'Mahal', 
              ratio = 'variable', 
              breaks = NULL)
\end{CodeInput}
\end{CodeChunk}

Continuing with the Lalonde example, next we'll estimate the effects
along the frontier with the \code{estimateEffects()} function, which
takes the output from \code{makeFrontier} to estimate the effect of
the treatment along all values of the frontier. This can be quite slow
for the obvious reason that if there exist thousands or tens of
thousands of points on the frontier, the code can be no faster than
the time it takes to estimate a single effect times the total number
of points (which quickly becomes quite long). Very soon, we will 
support estimating a random sample of points along the frontier to 
reduce computation time.

With the Lalonde example, the code is as follows. 

\begin{CodeChunk}
\begin{CodeInput}
my.estimates <- estimateEffects(my.frontier, 
                             're78 ~ treat')
\end{CodeInput}
\end{CodeChunk}

The first argument is the output from \code{makeFrontier} and the
second is the formula passed to the \code{lm()}
function. \code{estimateEffects} stores the estimates and the 95\%
confidence interval for each point it estimates.

Next, we can plot the frontier and the estimates with the plotting
functions, as follows. 

\begin{CodeChunk}
\begin{CodeInput}
# Plot frontier
plotFrontier(my.frontier)

# Plot estimates
plotEstimates(my.frontier, my.estimates)
\end{CodeInput}
\end{CodeChunk}

Both plotting functions use the ellipses feature to pass arguments to
the base \code{plot} function, so any argument that one can use with
\code{plot()} can also be used with \code{plotFrontier()} and
\code{plotEstimates()}, such as `main' for the title, `xlab' and
`ylab' for the axis labels, etc. 

Lastly, users may wish to export a data set on the frontier for 
additional analysis. To do so, use the \code{generateDataset()} 
function, as follows. 

\begin{CodeChunk}
\begin{CodeInput}
n = 1000 # Identify the point from which to select the data
generateDataset(my.frontier, N = n)
\end{CodeInput}
\end{CodeChunk}

If the estimand is variable ratio, as it is by default, the
exported data set will include the appropriate weights necessary
for estimating the FSATT.

\end{document}
